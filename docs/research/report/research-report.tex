%%%%%%%%%%%%%%%%%%%%%%%%%%%%%%%%%%%%%%%%%
% Academic Title Page
% LaTeX Template
% Version 2.0 (17/7/17)
%
% This template was downloaded from:
% http://www.LaTeXTemplates.com
%
% Original author:
% WikiBooks (LaTeX - Title Creation) with modifications by:
% Vel (vel@latextemplates.com)
%
% License:
% CC BY-NC-SA 3.0 (http://creativecommons.org/licenses/by-nc-sa/3.0/)
%
% Instructions for using this template:
% This title page is capable of being compiled as is. This is not useful for
% including it in another document. To do this, you have two options:
%
% 1) Copy/paste everything between \begin{document} and \end{document}
% starting at \begin{titlepage} and paste this into another LaTeX file where you
% want your title page.
% OR
% 2) Remove everything outside the \begin{titlepage} and \end{titlepage}, rename
% this file and move it to the same directory as the LaTeX file you wish to add it to.
% Then add \input{./<new filename>.tex} to your LaTeX file where you want your
% title page.
%
%%%%%%%%%%%%%%%%%%%%%%%%%%%%%%%%%%%%%%%%%

%----------------------------------------------------------------------------------------
%	PACKAGES AND OTHER DOCUMENT CONFIGURATIONS
%----------------------------------------------------------------------------------------

\documentclass[11pt]{article}

\usepackage[a4paper, margin={2cm, 3cm}]{geometry}

\usepackage{multicol}

\usepackage[utf8]{inputenc} % Required for inputting international characters
\usepackage[T1]{fontenc} % Output font encoding for international characters

\usepackage{mathpazo} % Palatino font

\usepackage[english]{babel}
\usepackage{csquotes}

\usepackage{fancyhdr}
\pagestyle{fancy}
\fancyhf{}
\rhead{D.J. Holland}
\lhead{ICT40010 — Research Report}
\rfoot{\thepage}

\usepackage[toc,page]{appendix}

\usepackage[
  backend=biber,
]{biblatex}
\addbibresource{../../ICT40010.bib}

\usepackage{graphicx}

\usepackage{minted}
\usemintedstyle{manni}
\newminted{rust}{
  autogobble,
  bgcolor=gray!10,
  fontsize=\footnotesize,
  samepage,
}
\newmintinline{rust}{bgcolor=gray!10}
\def\rust{\rustinline}
\newmintinline{text}{bgcolor=gray!10}
\def\code{\textinline}

\usepackage[dvipsnames]{xcolor}
\definecolor{FuchsiaPink}{HTML}{af30ed}
\definecolor{ScooterBlue}{HTML}{30c9ed}

\usepackage{hyperref}
\hypersetup{
    colorlinks=true,
    allcolors=FuchsiaPink,
}

\usepackage{fontspec}
\newfontface\emojifont{Twitter Color Emoji}
\newcommand{\emoji}[1]{{\emojifont{#1}}}

% \usepackage{pgfplots}

% \usepackage{filecontents}
% \begin{filecontents*}{data.csv}
%   min,max,avg
%   3333400,3508624,3339428
%   3333378,3503090,3339712
%   3333388,3394145,3339213
%   4956730,6405593,5561731
%   27910872,30017652,29159187
%   3330164,53269932,3591565
%   3333490,3485312,3339783
%   3333226,3479624,3339425
%   7929630,10788166,9064394
%   32618952,34970334,33390051
%   3333237,3478768,3338963
%   3333358,3371851,3338318
%   3333303,32700345,6513392
%   38573035,42248249,40159530
%   113376605,120017429,116978969
%   3333004,3368786,3338665
%   3333733,77577666,14687106
%   3341002,99495628,27497440
%   118402484,125087299,121400110
%   330656582,341476958,336928584
%   3332733,3524289,3342311
%   3333315,28709836,3590316
%   3340778,96427123,48864434
%   344543184,356602594,351093078
%   912672550,912672550,912672550
%   3332782,3348864,3338272
%   3333159,3374808,3339088
%   3332977,3538749,3342647
%   5933801,21524405,8229011
%   29196583,39673661,32725063
%   3333118,31476296,4781756
%   3333297,37010872,5053198
%   3332432,24817865,5770058
%   9063973,12260164,10678923
%   35112622,38120969,36422114
%   17260657,19470855,18280655
%   3336435,132907023,18332806
%   3333828,76838384,19099024
%   5600161,40768613,24198861
%   43936360,54033737,49376229
%   52678362,55236886,53914765
%   3333739,382217655,54188749
%   3350932,219124438,54695925
%   5920143,111353347,59538425
%   83091854,84955928,83933094
%   134695479,138703348,136766750
%   132944730,138898976,136180223
%   135582178,270626137,157031598
%   140975194,144129043,142376131
%   174880377,177765368,176125589
% \end{filecontents*}

\begin{document}

% I’m going to tell you thing → tell thing → I’ve told you thing.
% Shouldn’t be confused/surprised why something is being shown.

%----------------------------------------------------------------------------------------
%	TITLE PAGE
%----------------------------------------------------------------------------------------

\begin{titlepage} % Suppresses displaying the page number on the title page and the subsequent page counts as page 1
  \newcommand{\HRule}{\rule{\linewidth}{0.5mm}} % Defines a new command for horizontal lines, change thickness here

  \center{} % Centre everything on the page

  %------------------------------------------------
  %	Headings
  %------------------------------------------------

  % Main heading such as the name of your university/college
  \textsc{\LARGE Swinburne University of Technology}\\[1.5cm]

  % Major heading such as course name
  \textsc{
    \Large
    Bachelor of Engineering\\
    (Software Engineering)\\
    (Honours)
  }\\[0.5cm]

  % Minor heading such as course title
  \textsc{\large ICT40010 Research Report A}\\[0.5cm]

  %------------------------------------------------
  %	Title
  %------------------------------------------------

  \HRule{}\\[0.4cm]

  % Title of your document
  {\huge\bfseries Forward Rendering and Deferred Rendering: An Analysis of Real-Time Graphics Pipelines}\\[0.2cm]

  \HRule{}\\[1.5cm]

  %------------------------------------------------
  %	Author(s)
  %------------------------------------------------

  \begin{minipage}{0.4\textwidth}
    \begin{flushleft}
      \large
      \textit{Author}\\
      D.J. \textsc{Holland}
    \end{flushleft}
  \end{minipage}
  {~}
  \begin{minipage}{0.4\textwidth}
    \begin{flushright}
      \large
      \textit{Supervisors}\\
      Dr.\ Clinton \textsc{Woodward}\\
      Dr.\ Charlotte \textsc{Pierce}
    \end{flushright}
  \end{minipage}

  % If you don't want a supervisor, uncomment the two lines below and comment the code above
  %{\large\textit{Author}}\\
  %John \textsc{Smith} % Your name

  %------------------------------------------------
  %	Date
  %------------------------------------------------

  \vfill\vfill\vfill % Position the date 3/4 down the remaining page

  {\large\today} % Date, change the \today to a set date if you want to be precise

  %------------------------------------------------
  %	Logo
  %------------------------------------------------

  \vfill\vfill
  \includegraphics[width=0.2\textwidth]{../../swin-logo.jpg}\\[1cm] % Include a department/university logo - this will require the graphicx package

  %----------------------------------------------------------------------------------------

  % \vfill % Push the date up 1/4 of the remaining page

\end{titlepage}

%----------------------------------------------------------------------------------------
%	ABSTRACT
%----------------------------------------------------------------------------------------

\pagenumbering{gobble}

\begin{abstract}
  % - Domain/context
  % - gap
  % - Method
  % - Results/outcomes
  % - Implications
  TODO: abstract
\end{abstract}

\newpage

%----------------------------------------------------------------------------------------
%	CONTENTS
%----------------------------------------------------------------------------------------

\tableofcontents
% \pagenumbering{gobble}
\newpage
\pagenumbering{arabic}

%----------------------------------------------------------------------------------------
%	REPORT
%----------------------------------------------------------------------------------------

\section{Introduction}
% - Restate domain / gap
% - Purpose of report. Present-tense
% - Purpose of research. Past-tense
% - Background
%     - Reference TR
%     - Screenshot?
TODO: intro

% \begin{tikzpicture}
%   \begin{axis}
%     \addplot[
%       YellowGreen,
%       only marks,
%     ]
%     table[
%         col sep=comma,
%         x expr=\coordindex + 1,
%         y=min,
%       ] {data.csv};
%     \addplot[
%       Orange,
%       only marks,
%     ]
%     table[
%         col sep=comma,
%         x expr=\coordindex + 1,
%         y=max,
%       ] {data.csv};
%     \addplot[
%       Cerulean,
%       only marks,
%       error bars/.cd,
%       y dir=both,
%       y explicit,
%     ]
%     table[
%         col sep=comma,
%         x expr=\coordindex + 1,
%         y=avg,
%         y error plus expr=\thisrow{max} - \thisrow{avg},
%         y error minus expr=\thisrow{avg} - \thisrow{min},
%       ] {data.csv};
%     \addplot[mark=none] {(1/60)*1000000000};
%   \end{axis}
% \end{tikzpicture}

\section{Method}
% - Restate question
%     - What is the measurable difference between forward/deferred?
% - Justify configuration
%     - How parameter changes help answer question
%     - What are my assumptions?
% - Configuration details
%     - Constants? But (small) why?
%     - Parameter sequence
%     - Ranges (appendix?)
%     - Example table
%         - Screenshot of test run

TODO: method

\section{Results}
% - Results (facts/data)
%     - Small words on data.
%         - Answer the question(s)
%         - Empirical evidence
%         - No speculation, only observation
%     - MEGA CHART (landscape)
%     - Other charts, e.g.
%         - Variation from avg
%     - Full data table in appendices

TODO: results

\section{Discussion}
% - Discussion (opinions)
%     - About Data
%     - About method
%     - About THE FUTURE
%         - Backlog teuxdeux list

TODO: discussion

\section{Conclusion}
% - Conclusion
%     - No new, only restate (dog meme)
%     - Emphasise the main outcome/implication

TODO: conclusion

\section{Acknowledgements}

TODO: acknowledgements

\newpage

%----------------------------------------------------------------------------------------
%	BIBLIOGRAPHY
%----------------------------------------------------------------------------------------

\printbibliography[heading=bibintoc]{}

\newpage

%----------------------------------------------------------------------------------------
%	APPENDICES
%----------------------------------------------------------------------------------------

\begin{appendices}

\end{appendices}

\end{document}
